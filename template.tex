\documentclass[11pt, a4paper, sans, colorlinks]{moderncv}
\usepackage[utf8]{inputenc}
\usepackage[T1]{fontenc}

\moderncvstyle{classic}
\moderncvcolor{purple}

\usepackage[scale=0.75]{geometry}
 
\name{Maciej}{Błędkowski}
%\title{Curriculum Vitae}
\phone[mobile]{+48 725 033 880}
\email{m@mble.dk}
\social[github]{mbledkowski}
\social[linkedin]{mbled}
\social[instagram][instagram.com/hqpixl]{ig/hqpixl}

\begin{document}

\definecolor{links}{HTML}{000000}
\hypersetup{urlcolor=links}

\makecvtitle

\newcommand{\book}[2]{\cvlistitem{\textbf{#1}#2}}
\newcommand{\course}[2]{\cvlistitem{\textbf{#1}#2}}

\newcommand{\tech}[1]{\vskip -2mm  \textbf{Skills:} \ignorespaces #1 \ignorespaces \\}
\newcommand{\responsibilities}[1]{\vskip -4mm \textbf{Responsibilities:} \ignorespaces #1 \ignorespaces \\}

\newcommand{\skill}[1]{#1,}
\newcommand{\fullskill}[1]{#1,}

\section{Skills and Technologies}

\cvitem{Frontend}{\textbf{TypeScript}, \textbf{Nuxt}, \textbf{Vue.js}, Vuex, Next, React, \textbf{CSS}, \textbf{TailWindCSS} Vite, Webpack, Babel, Eslint, Jest}

\cvitem{Backend}{\textbf{TypeScript}, \textbf{Express.js}, \textbf{Nest.js}, \textbf{Node.js}, \textbf{Deno}, Bun, \textbf{Python}, Rust, C++, C\#, Tensorflow, PyTorch, PostgreSQL, MongoDB, Git}

\cvitem{Infrastructure}{\textbf{Linux}, \textbf{Docker}, \textbf{Docker Compose}, \textbf{Ubuntu/Debian}, \textbf{Fedora/RHEL}, \textbf{Arch Linux}, \textbf{NixOS}, GCI, Oracle Cloud, \textbf{Netlify}, Heroku, Firebase, Supabase}

\cvitem{Design}{\textbf{Figma}, \textbf{Affinity Designer}, \textbf{Affinity Photo}, Adobe Photoshop, DaVinci Resolve}

\section{Projects}

\cventry{2020}{\href{https://github.com/ajayyy/SponsorBlock/graphs/contributors}{SponsorBlock popup window}}{Affinity/HTML/SCSS/TypeScript}{}{}
{Redesigned GUI for broadly used Chrome/Firefox extension. SponsorBlock is an extension whose main focus is on skipping advertisement sections of YouTube videos. \href{https://github.com/ajayyy/SponsorBlock}{https://github.com/ajayyy/SponsorBlock/}}

\cventry{2022}{\href{https://github.com/hspsh/syncronium}{Syncronium's twitter integration}}{TypeScript/Twitter API}{}{}
{Created Twitter integration for Syncronium bot. Syncronium is a server application for synchronising events between different platforms.}

\cventry{2022}{\href{https://github.com/mbledkowski/antipixel-website}{Antipixel archive}}{CSS/Fresh/Preact/TypeScript/Deno}{}{}
{Archive of "antipixels" - mini web badges used on webpages in 00'. \href{https://antipixel.art}{antipixel.art}}

\cventry{2020-2022}{\href{https://github.com/mbledkowski/dnidomatury}{Dni do matury}}{Affinity/SCSS/Nuxt/Vue.js/Day.js/TypeScript/Jest}{}{}
{Application with informations about matura exam. \href{https://dnidomaturypl.netlify.app}{dnidomaturypl.netlify.app}}

\cventry{2022}{\href{https://github.com/oneacik/stampli}{Stampli}}{Affinity/SCSS/React/TypeScript}{}{}
{Application for collecting loyalty stamps. \href{https://github.com/ksidelta/stampli}{https://github.com/ksidelta/stampli}}

\cventry{2022}{\href{https://gitlab.com/SimLE/simle-website}{SimLE website}}{Affinity/HTML/CSS/Hugo}{}{}
{Webpage for Politechnika Gdańska's science club - SimLE. \href{https://simle2.netlify.app}{simle2.netlify.app} \href{https://simle.pl}{simle.pl(work in progress)} \href{https://gitlab.com/SimLE/simle-website}{https://gitlab.com/SimLE/simle-website}}

\cventry{2021-2022}{\href{https://github.com/mbledkowski/keycomp}{Keycomp}}{SCSS/Vue.js/Vuex/Chart.js/TypeScript/Python/Node.js/PostgreSQL}{}{}
{Website with technical details about mechanical keyboard switches. \href{https://keycomp.co}{keycomp.co} \href{https://github.com/mbledkowski/keycomp}{https://github.com/mbledkowski/keycomp}}

\cventry{2021-2022}{\href{https://github.com/hspsh/dcr}{Dynamic Code Renderer}}{CSS/React/Express.js/Knex/TypeScript/Node.js/Docker}{}{}
{Web application for rendering Graphviz's dot files \href{https://github.com/hspsh/dcr}{https://github.com/hspsh/dcr}}

\cventry{2021}{\href{https://github.com/mbledkowski/mble.dk}{mble.dk}}{SCSS/Vue.js/Vuex/TypeScript}{}{}
{My personal portfolio/blog website. \href{https://mble.dk}{mble.dk} \href{https://github.com/mbledkowski/mble.dk}{https://github.com/mbledkowski/mble.dk}}

\cventry{2018-2019}{\href{https://youtube.com/@hq2659}{YouTube - HQ265}}{DaVinci Resolve/Affinity Designer}{}{}
{YouTube channel, where I present Computer Science topics in easily digestible manner.}

\pagebreak

\section{Education}

\cventry{2021.03 - now}{WSB Merito in Gdańsk}{Computer Science degree, Front-End specialisation}{}{}
{Student of the second year at WSB Universities}

\section{Language}

\cventry{}{English}{C1}{}{}
{I know English well enough to be able to read technical documentation and communicate with other people. My score on the advanced level English language matura exam was 89\%. }

\cventry{}{Polish}{C2 (native)}{}{}{}

\section{Presentations and Workshops}

\cventry{2022.11}{Hackerspace Pomorze stand}{Experyment Gdynia}
{}{}
{A stand presenting different technology projects from Hackerspace Pomorze.}

\cventry{2021.09}{Linux Party}{Hackerspace Pomorze}
{}{}
{A presentation about Desktop Environments and Window Managers available for Linux.}

\section{Courses}

\course{Clean Code}{ book by Robert C. Martin}
\course{You Don't Know JS}{ book series by Kyle Simpson}
\course{Introduction to Vue 3}{ on FrontendMasters by Sarah Drasner}
\course{Learn React}{ on Scrimba}
\course{HTML/CSS/JS}{ on Sololearn}

\section{Organisations}

\cventry{2019.08 - now}{Hackerspace Pomorze / Hackerspace Trójmiasto}{}{}{}
{
Spaces for people interested in technology.\\
Place where tech-savvy people can meet together and exchange their knowledge about variety of topics.
}

\section{GDPR}

\cventry{}{English}{}{}{}
{I hereby give consent for my personal data to be processed for the purpose of conducting recruitment for the position for which I am applying, and for the purposes of any future recruitment processes.}

\cventry{}{Polski}{}{}{}
{Wyrażam zgodę na przetwarzanie moich danych osobowych w celu prowadzenia rekrutacji na aplikowane przeze mnie stanowisko, jak i na potrzeby przyszłych rekrutacji.}

\end{document}
